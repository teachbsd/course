% Included from both -slides and -handout versions.
\documentclass[pdftex]{beamer} % used to trigger beamer mode in Emacs,
                               % normally commented out.

\usetheme{metropolis}

\usepackage[english]{babel}
\usepackage[latin1]{inputenc}
\usepackage{graphicx}
\usepackage{times}
\usepackage[T1]{fontenc}
\usepackage{fancyvrb}
\usepackage{listings}
\begin{document}
\lstset{language=C, escapeinside={(*@}{@*)}, numbers=left,
  basicstyle=\tiny, showspaces=false, showtabs=false}

\title{Introduction to Operating Systems}
\subtitle{Through tracing, analysis, and experimentation}
%\institute{University of Cambridge}
\author{George V. Neville-Neil}
%\author{Dr Robert N. M. Watson}
\date{1 August 2016}

\begin{frame}
  \titlepage
\end{frame}

\section{Storing Information}
\label{sec:storage}

\subsection{Naming of Names}
\label{sec:naming}

\begin{frame}
  \frametitle{File Systems Overview}
  \begin{description}
  \item[naming] Translating human names to usable objects
  \item[storage] Store and retrieve blocks of data
  \end{description}
\end{frame}

\begin{frame}[fragile]
  \frametitle{Naming}
  \begin{itemize}
  \item Translate a human name to something
  \item \verb+namei+ is the main interface
  \item All names reside in the name cache
  \end{itemize}
\end{frame}

\begin{frame}[fragile]
  \frametitle{Name Lookup}
  \begin{itemize}
  \item What names are being looked up?
  \end{itemize}
\begin{lstlisting}
dtrace -n 'vfs:namei:lookup:entry { printf("%s", stringof(arg1));}'
CPU     ID                    FUNCTION:NAME
  2  27847                     lookup:entry /bin/ls
  2  27847                     lookup:entry /libexec/ld-elf.so.1
  2  27847                     lookup:entry /etc
  2  27847                     lookup:entry /etc/libmap.conf
  2  27847                     lookup:entry /etc/libmap.conf
\end{lstlisting}
\end{frame}

\begin{frame}
  \frametitle{Name Cache}
  \begin{itemize}
  \item Speeds up searching
  \item Maintains positive and negative results
  \item Invalidation on changes in directories
  \end{itemize}
\end{frame}

\begin{frame}[fragile]
  \frametitle{Who is missing the cache?}
\begin{lstlisting}
dtrace -n 'vfs:namecache:lookup:miss { printf("%s", stringof(arg1));}'
\end{lstlisting}
\end{frame}

\begin{frame}[fragile]
  \frametitle{Name Cache Module}
  \begin{description}
  \item[enter] Add a positive entry
  \item[enter\_negative] Add a negative entry
  \item[lookup:hit] Name found in positive cache
  \item[lookup:hit-negative] Name found in negative cache
  \item[lookup:miss] Name not found in cache
  \item[purge] Remove positive entry
  \item[purge\_negative] Remove negative entry
  \item[zap] Remove positive entry with or without vnode
  \item[zap\_negative] Remove negative entry with or without vnode
  \end{description}
\end{frame}

\begin{frame}[fragile]
  \frametitle{Adding negative entries}
\begin{lstlisting}
 dtrace -n 'vfs:namecache:enter_negative: { printf("%s", stringof(arg1)); }'
\end{lstlisting}
\end{frame}


\subsection{Organizing Data}
\label{sec:organization}

\begin{frame}[fragile]
  \frametitle{VNODE Operations}
  \begin{itemize}
  \item After a path or name is looked up
  \item Do something with a vnode
  \item \verb|open|, \verb|close|, \verb|read|, \verb|write|
  \end{itemize}
\end{frame}

\subsection{Data on Disk}
\label{sec:disk}

\begin{frame}
  \frametitle{GEOM Oveview}
  \begin{itemize}
  \item Framework for transforming storage requests
  \item Object Oriented
  \item Shuttles I/O requests between filesystems and devices
  \end{itemize}
\end{frame}

\begin{frame}
  \frametitle{GEOM Layers}
  \begin{description}
  \item[MBR] Master Boot Record
  \item[BSD] BSD Slice
  \item[ELI] Encryption
  \item[MIRROR] Disk mirroring
  \item[JOURNAL] Journaling
  \item[RAID] Software RAID
\end{description}
\end{frame}

\begin{frame}
  \frametitle{A Visual Example}
  
\end{frame}

\begin{frame}
  \frametitle{Performing I/O}
sudo dtrace -n ::g\_io\*:entry  

This gets us the most likely 8 routines we care about.

\end{frame}

\end{document}
%%% Local Variables:
%%% mode: latex
%%% TeX-master: "lecture4-storage"
%%% End:
