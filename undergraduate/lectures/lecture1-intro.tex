% Included from both -slides and -handout versions.
\documentclass[pdftex]{beamer} % used to trigger beamer mode in Emacs,
                               % normally commented out.

\usetheme{metropolis}

\usepackage[english]{babel}
\usepackage[latin1]{inputenc}
\usepackage{graphicx}
\usepackage{times}
\usepackage[T1]{fontenc}
\usepackage{fancyvrb}
\usepackage{listings}
\begin{document}
\lstset{language=C, escapeinside={(*@}{@*)}, numbers=left,
  basicstyle=\tiny, showspaces=false, showtabs=false}

\title{Introduction to Operating Systems}
\subtitle{Through tracing, analysis, and experimentation}
%\institute{University of Cambridge}
\author{George V. Neville-Neil}
%\author{Dr Robert N. M. Watson}
\date{1 August 2016}

\begin{frame}
  \titlepage
\end{frame}

\section{Course Overview}

\begin{frame}
  \frametitle{Course Goals}
  \begin{itemize}
  \item 
  \end{itemize}
\end{frame}

\begin{frame}
  \frametitle{Course Outline}
  \begin{description}
  \item[Monday] 
  \item[Tuesday]
  \item[Wednesday] 
  \item[Thursday] 
  \item[Friday] 
  \end{description}
\end{frame}

\begin{frame}
  \frametitle{Lab Work}
  \begin{description}
  \item []
  \end{description}
\end{frame}

\section{Example System}
\label{sec:example-system}

\begin{frame}
  \frametitle{Our Challenge}
  \begin{itemize}
  \item How does data move in a real world system?
  \item Serving Data from Storage to Clients
    \begin{description}
    \item[Program] Web Server
    \item[Stored Data] Filesystem
    \item[Communication] Network Stack
    \end{description}
  \end{itemize}
\end{frame}

\begin{frame}
  \frametitle{What do computers do?}
  \begin{description}
  \item[Processing] Converting information from one form to another.
  \item[Communication] Moving information between one or more systems.
  \item[Storage] Maintaining information over time.
  \end{description}
\end{frame}

\begin{frame}
  \frametitle{What is an Operating System?}
  \begin{center}
    \begin{itemize}
    \item \emph{White Boarding Exercise.}
    \end{itemize}
  \end{center}
\end{frame}

\begin{frame}
  \frametitle{What an Operating System Does}
  \begin{itemize}
  \item Provides the \emph{Programming Model}
  \item Protects Programs from each other
  \item Controls access to hardware
  \item Ensures fair sharing of resources
  \end{itemize}
\end{frame}

\begin{frame}
  \frametitle{Special Considerations}
  \begin{itemize}
  \item An OS kernel is one, large program.
  \item More than $50,000$ functions.
  \item $5677$ files    
  \item $5,131,552$ lines of C code.
  \item A unique programming environment
  \end{itemize}
\end{frame}

\begin{frame}
  \frametitle{Design and Implementation Requirements}
  \begin{description}
  \item[Fast]
  \item[Safe] 
  \item[Flexible] 
  \end{description}
\end{frame}

\begin{frame}[fragile]
  \frametitle{User vs. Kernel Space}
  \begin{columns}[t]
    \begin{column}{5cm}
      \emph{User Space}
      \begin{itemize}
      \item 
      \end{itemize}
    \end{column}
    \begin{column}{5cm}
      \emph{Kernel Space}
      \begin{itemize}
      \item 
      \end{itemize}
    \end{column}
  \end{columns}
\end{frame}

\begin{frame}
  \frametitle{System Calls: The Operating System's API}
  
\end{frame}

\section{Introduction to Tracing}
\label{sec:intro-tracing}



\end{document}

%%% Local Variables:
%%% mode: latex
%%% TeX-master: t
%%% End:
