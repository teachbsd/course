% Included from both -slides and -handout versions.
\documentclass[pdftex]{beamer} % used to trigger beamer mode in Emacs,
                               % normally commented out.
\usetheme{metropolis}

\usepackage[english]{babel}
\usepackage[latin1]{inputenc}
\usepackage{graphicx}
\usepackage{times}
\usepackage[T1]{fontenc}
\usepackage{fancyvrb}
\usepackage{listings}
\begin{document}
\lstset{language=C, escapeinside={(*@}{@*)}, numbers=left,
  basicstyle=\tiny, showspaces=false, showtabs=false}

\title{Introduction to Operating Systems}
\subtitle{Through tracing, analysis, and experimentation}
%\institute{University of Cambridge}
\author{George V. Neville-Neil}
%\author{Dr Robert N. M. Watson}
\date{1 August 2016}

\begin{frame}
  \titlepage
\end{frame}

\section{Processes}
\label{sec:processes}

\begin{frame}
  \frametitle{The process model}

  \begin{enumerate}
    \item The process model and its evolution
    \item An introduction to virtual memory
    \item Where do programs come from?
    \item Traps and system calls
    \item Reading for next time
  \end{enumerate}
\end{frame}

\begin{frame}
  \frametitle{Review: What is an Operating System?}
  \begin{itemize}
  \item Provider of useful abstractions
  \item Generic interface to varied hardware resources
  \item Controller of resources
  \item Provider of basic security
  \end{itemize}
\end{frame}

\begin{frame}
  \frametitle{Before Processes and Virtual Memory}
  \begin{itemize}
  \item All code in a single address space
  \item No memory protection
  \item Each program must cooperate with all others
  \item Core wars
  \end{itemize}
\end{frame}

\begin{frame}
  \frametitle{What is a process?}
  \begin{itemize}
  \item Container for code
  \item Protective shell between competing programs
  \item \emph{The} defining abstraction on which all modern computing rests
  \end{itemize}
\end{frame}

\begin{frame}
  \frametitle{Process Contents}
  \begin{itemize}
  \item 
  \end{itemize}
\end{frame}

\begin{frame}
  \frametitle{Process Lifecycle}
  
\end{frame}

\begin{frame}
  \frametitle{Scheduling}
  
\end{frame}


\section{The Illusion of Memory}
\label{sec:memory}

\end{document}


%%% Local Variables:
%%% mode: latex
%%% TeX-master: "lecture2-processes.tex"
%%% End:
